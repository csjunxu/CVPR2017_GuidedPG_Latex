\documentclass[10pt,twocolumn,letterpaper]{article}

\usepackage{cvpr}
\usepackage{times}
\usepackage{epsfig}
\usepackage{graphicx}
\usepackage{amsmath}
\usepackage{amssymb}
\usepackage{subfigure}
\usepackage{upgreek}

% Include other packages here, before hyperref.

% If you comment hyperref and then uncomment it, you should delete
% egpaper.aux before re-running latex.  (Or just hit 'q' on the first latex
% run, let it finish, and you should be clear).
\usepackage[pagebackref=true,breaklinks=true,letterpaper=true,colorlinks,bookmarks=false]{hyperref}

% \cvprfinalcopy % *** Uncomment this line for the final submission

\def\cvprPaperID{****} % *** Enter the CVPR Paper ID here
\def\httilde{\mbox{\tt\raisebox{-.5ex}{\symbol{126}}}}

% Pages are numbered in submission mode, and unnumbered in camera-ready
\ifcvprfinal\pagestyle{empty}\fi
\begin{document}

%%%%%%%%% TITLE
\title{External Patch Group Prior Guided Internal Subspace Learning for Real Image Denoising}

\author{First Author\\
Institution1\\
Institution1 address\\
{\tt\small firstauthor@i1.org}
% For a paper whose authors are all at the same institution,
% omit the following lines up until the closing ``}''.
% Additional authors and addresses can be added with ``\and'',
% just like the second author.
% To save space, use either the email address or home page, not both
\and
Second Author\\
Institution2\\
First line of institution2 address\\
{\tt\small secondauthor@i2.org}
}

\maketitle

%%%%%%%%% ABSTRACT
\begin{abstract}
Existing image denoising methods largely depends on noise modeling and estimation. The commonly used noise models, additive white Gaussian, are inflexible in describing the complex noise on real noisy images. This would limit the performance of existing methods on denoising real noisy images. In this paper, we firstly demonstrate that almost all state-of-the-art methods on removing Gaussian noise and real noise are limited in denoising real noisy images. We demonstrate that a simple Patch Group based Prior Learning model on RGB images can achieve better performance than existing denoising methods, especially the ones designed for real noise in natural images. Besides, we employ the external patch group prior learning for internal clustering and subspace learning. This external inormation guided internal denoising methods acheives even better than the external PG prior based methods and the fully internal PG prior based method. Through extensive on standard datasets on real noisy images with groundtruth, we demonstrate that the proposed method achieves much better denoising performance than the other state-of-the-art methods on Gaussian noise removal and real noise removal. 
\end{abstract}

%%%%%%%%% BODY TEXT
\section{Introduction}
Image denoising is a fundermental problem in computer vision and image processing. It is an ideal platform for testing natural image models and provides high-quality images for other conputer vision tasks such as image registration, segmentation, and pattern recognition, etc. For several decades, there emerge numerous image denoising methods \cite{nlm,foe,ksvd,bm3d,lssc,epll,burger2012image,wnnm,csf,pgpd,chen2015learning}, and all of them focus mainly on dealing with additive white Gaussian noise (AWGN). Recently, several discriminative learning methods \cite{burger2012image,csf,chen2015learning} achieving expressive performance on Gaussian noise removal. These methods require a set of paired images, namely clean ground-truth images and the simulated noisy counterparts degraded by identical noise (mainly additive white Gaussian noise, AWGN), to learn an effective model for image denoising. However, the noise in real images are much more complex than Gaussian, since it depends on camera series, brands, as well as the settings (ISO, shutter speed, and aperture, etc). Thus, the model learned with AWGN would become much less effective for denoising real noisy images. What's more, usually real noisy images do not have clean counterparts. Therefore, almost all current discriminative learning methods cannot be directly applied to real noisy images.

In the last decade, the methods of \cite{fullyblind,rabie2005robust,Liu2008,almapg,noiseclinic,Zhu_2016_CVPR,crosschannel2016} are designed to deal with real noisy images. Almost all these methods coincidently employ a two-stage framework: in the first stage, assuming a distribution model (usually Gaussian) on the noise and estimate its parameters; in the second stage, performing denoising with the help of the noise modeling and estimation in the first stage. However, the Gaussian assumption is inflexible in describing the complex noise on real noisy images \cite{Liu2008,crosschannel2016}. Although the mixture of Gaussians (MoG) model is possible to approximate any unknown noise \cite{Zhu_2016_CVPR}, estimating its parameters is often time consuming via nonparametric Bayesian techniques \cite{Zhu_2016_CVPR} \cite{Bishop}. 

The above mentioned limitations indicate that, novel denoising methods are been waiting for which can: 1) avoid noise modeling and estimation; 2) deal with complex noise on real noisy images. In this paper, we attempt to solve the two problems in an integrated way for robust real image denoising. We find that the Patch Group Prior learning based denoising \cite{PGPD} method learned on natural clean RGB images are enough to outperform the above mentioned denoising methods. We also propose a fully internal PG prior based denoising method which achieve better performance than the fully external method. Most importantly, we found that the external PG prior guided internal method can achieve even better and faster performance on real image denoising. The external PG prior learning based model is employed to guide the clustering of internal PGs extracted from the input noisy images. Then for each cluster of PGs, we perform subspace learning by PCA and denoising by weighted sparse coding. The noise level of the noisy images is inheriently expressed in the singular values, which can be used as weights for the sparse coding. We perform comprehensive experiments on real noisy images captured by different CMOS or CCS sensors. The results demonstrate that our method achieves comparable or even better performance on denoising real noisy images. This reveals the potential advantages of combining external and internal information o natural images on robust and complex real noisy image denoising problem.

%The in-camera imaging pipeline usually includes image demosaicing, white balance and color space transform, gamut mapping, tone mapping, and JPEG comression \cite{NewInCamera,crosschannel2016}.

\subsection{Our Contributions}
The contributions of this paper are summarized as follows:
\begin{itemize}
\item We propose a noval method which combine the external and internal PG prior for real noisy image denoising problem;
\item Our method doesn't need noise modeling and estimation, and the noise levels of real noisy images are automatically expresssed by the singular values of learned subspace;
\item We achieve much better performance, visual quality, PSNR, SSIM, and speed for real image denoising problem.
\end{itemize}

\section{Related Work}
\subsection{Patch Group Prior of Natural Images}
Patch prior is an approximation of the prior of natural images. There are several famous work on this area such as the K-SVD algorithm \cite{ksvd}. The seminar work on patch prior modeling is the expected patch log likelihood (EPLL), which models the space of natural image patches via Gaussian Mixture Model. Recently, the Patch Group prior \cite{pgpd} is proposed to directly model the non-local self similarity within natural images. The patch group prior demonstrate better properties via better image denoising performance on natural images. However, PGPD only utilizes the information of clean natural images, but not fully make full of the NSS information of noisy input images. In this paper, we combine the information of external natural clean images and internal real noisy images to achieve adaptive patch group prior modeling. In fact, we use the external Pacth Group prior to guide the structural clustering of internal patch groups in real noisy images. For each cluster of PGs, we propose to use PCA for subspace learning. The eigenvectors are the basis of this subspace and hence can be used as an orthonormal dictionary. The eigenvalues reflect the information describing the noise levels and hence can be used as parameters for image denoising. Noted that the dictionary and parameters are obtained via a fully unsupervised way by subspace learning. Therefore, our model is highly adaptive to the noisy input images.

\subsection{Real Image Denoising}
To the best of our knowledge, the study of real image denoising can be dated back to the BLS-GSM model \cite{blsgsm}, in which Portilla et al. proposed to use scale mixture of Gaussian in overcomplete oriented pyramids to estimate the latent clean images. In \cite{fullyblind}, Portilla proposed to use a correlated Gaussian model for noise estimation of each wavelet subband. Based on the robust statistics theory \cite{huber2011robust}, the work of Rabie \cite{rabie2005robust} modeled the noisy pixels as outliers, which could be removed via Lorentzian robust estimator. In \cite{Liu2008}, Liu et al. proposed to use 'noise level function' (NLF) to estimate the noise and then use Gaussian conditional random field to obtain the latent clean image. Recently, Gong et al. proposed an optimization based method \cite{almapg}, which models the data fitting term by weighted sum of $\ell_{1}$ and $\ell_{2}$ norms and the regularization term by sparsity prior in the wavelet transform domain. Later, Lebrun el al. proposed a multiscale denoising algorithm called 'Noise Clinic' \cite{noiseclinic} for real image denoising task. This method generalizes the NL-Bayes \cite{nlbayes} to deal with signal, scale, and frequency dependent noise. Recently, Zhu et al. proposed a Bayesian model \cite{Zhu_2016_CVPR} which approximates the noise via Mixture of Gaussian (MoG) model \cite{Bishop}. The clean image is recovered from the noisy image by the proposed Low Rank MoG filter (LR-MoG). However, noise level estimation is already a challenging problem and denoising methods are quite sensitive to this parameter. Moreover, these methods are based on shrinkage models that are too simple to reflect reality, which results in over-smoothing of important structures such as small-scale text and textures. 

\section{Patch Group Prior Guided Internal Subspace Learning}
In this section, we formulate the framework of external Patch Group prior guided internal subspace learning. We first introduce the patch group prior leaning on clean natural RGB images. Then we formulate the external guided internal subspace learning. Finally, we discuss the differences between external subspaces and the corresponding internal subspace.

\subsection{External Patch Group Prior Learning on Clean Natural Images}
Images often demonstrate highly nonlocal self-similarity (NSS) property, which refers to the fact that a patch always have similar patches to it around the image. This property is a key successful factor in image denoising methods \cite{nlm,bm3d,lssc,ncsr,wnnm} and restoration methods \cite{}. In \cite{pgpd}, the NSS property is recently learned as an external prior in a patch group manner. In this section, we formulate the Patch Group prior on clean natural images.

In external PG prior learning on clean natural images, a PG is obtained via finding the $M$ most similar nonlocal patches to the local patch (size: $p\times p \times 3$ for RGB channels) in a given clean image. The similarity is measured by Euclidean distance or other distance measurements. In this work, we find similar PGs through the Euclidean distance based block matching in a large enough local window of size $W\times W$. The PG is denoted by $\{\mathbf{x}_{m}\}_{m=1}^{M}$, where $\mathbf{x}_{m}\in \mathbb{R}^{3p^{2}\times1}$ is a patch vector. The mean vector of this PG is $\boldsymbol{\upmu}=\frac{1}{M}\sum_{m=1}^{M}\mathbf{x}_{m}$, and $\mathbf{\overline{x}}_{m}=\mathbf{x}_{m}-\boldsymbol{\upmu}$ is the group mean subtracted patch vector. We call
\begin{equation}\label{equ1}
\setlength{\abovedisplayskip}{2pt}
\setlength{\belowdisplayskip}{2pt}
\mathbf{\overline{X}}\triangleq \{\mathbf{\overline{x}}_{m}\}, m=1,...,M
\end{equation}
the group mean subtracted PG, and it will be used to learn the NSS prior in our work. All these are similar with the definitions in \cite{pgpd}. 

From a given set of natural images, we can extract $N$ PGs, and  PG as
\begin{equation}\label{equ2}
\setlength{\abovedisplayskip}{2pt}
\setlength{\belowdisplayskip}{2pt}
\mathbf{\overline{X}}_{n}\triangleq \{\mathbf{\overline{x}}_{n,m}\}_{m=1}^{M}, n=1,...,N.
\end{equation}
We employ the patch group based Gaussian Mixture Model (PG-GMM) for NSS prior learning. The learning process is the same with that in \cite{pgpd}.

With PG-GMM, we aim to learn a set of $K$ Gaussians $\{\mathcal{N}(\boldsymbol{\upmu}_{k},\mathbf{\Sigma}_{k})\}$ from $N$ training PGs $\{\mathbf{\overline{X}}_{n}\}$, while requiring that all the $M$ patches $\{\mathbf{\overline{x}}_{n,m}\}$  in PG $\mathbf{\overline{X}}_{n}$ belong to the same Gaussian component and assume that the patches in the PG are independently sampled. Note that such an assumption is commonly used in patch based image modeling \cite{ksvd,lssc}. Then, the likelihood of $\{\mathbf{\overline{X}}_{n}\}$ can be calculated as
\begin{equation}\label{equ3}
\setlength{\abovedisplayskip}{3pt}
\setlength{\belowdisplayskip}{3pt}
P(\mathbf{\overline{X}}_{n})  = \sum\nolimits_{k=1}^{K}\pi_{k}\prod\nolimits_{m=1}^{M}\mathcal{N}(\mathbf{\overline{x}}_{n,m}|\boldsymbol{\upmu}_{k},\mathbf{\Sigma}_{k}).
\end{equation}
By assuming that all the PGs are independently sampled, the overall objective log-likelihood function is
\begin{equation}\label{equ4}
\setlength{\abovedisplayskip}{3pt}
\setlength{\belowdisplayskip}{3pt}
\begin{split}
\ln\mathcal{L}=\sum_{n=1}^{N} \ln(\sum_{k=1}^{K}\pi_{k}\prod_{m=1}^{M}\mathcal{N}(\mathbf{\overline{x}}_{n,m}|\boldsymbol{\upmu}_{k},\mathbf{\Sigma}_{k})).
\end{split}
\end{equation} 
We maximize the above objective function for PG-GMM learning. Finally, we obtain the GMM model with three sets of parameters including mixture weights $\{\pi_{k}\}_{k=1}^{K}$, mean vectors $\{\boldsymbol{\upmu}_{k}=\mathbf{0}\}_{k=1}^{K}$, and covariance matrices $\{\mathbf{\Sigma}_{k}\}_{k=1}^{K}$. Noted that in PGPD \cite{pgpd}, the mean vector of each cluster is natural zeros, i.e., $\boldsymbol{\upmu}_{k}=\mathbf{0}$.

\subsection{External PG Prior Guided Internal Subspace Learning}


\subsection{Comparing External PG Prior and Guided Internal Subspace Learning}


\section{The Overall Algorithm}

\subsection{Pair Sample Construction from Unpaired Samples}
In cross style transfer methods such as CDL and SCDL, the authors assume that the two different styles have paired data, i.e., for each data sample in one style, we can find paired data sample in the other style. However, in real world, the data from two different sources may be unpaired. For example, the real noisy images should not have groundtruth clean images of the same scene. The real low-resolution images should not have corresponding high-resolution images in the real world. The real blurry images should not have corresponding clear and high quality images in real world.

To deal with unpaired data, we could collect real noisy images and clean natural images from two different sources. The real noisy images are from the example images (18 images) of the Neat Image website while the clean natural images are from the training set (200 images) of the Berkeley Segmentation Dataset (BSDS500). To make use of the unpaired data samples, we employ searching strategy to construct the training dataset. That is, for each noisy image patch, we utilize the k-Nearest Neighbor (k-NN) algorithm to find the most similar patch in the clean images as the paired groundtruth patch. The similarity is measured by the Euclidean distance (also called squared error or $\ell_{2}$ norm).

\subsection{Structual Clustering and Model Selection}
In fact, different image structures should have different influences on dictioanry as well as the mapping function. Patches with flat region should have low rank structure within dictionary elements and identity mapping between noisy and latent clean patches. Patches with complex details should have more comprehensive dictionary elements within dictionary elements and more complex mapping function between noisy and clean patches. A single mapping function cannot deal with all these complex relationships. Hence, a structual clustering procedure is needed for complex solution. In this paper, we propose to employ Gaussian Mixture Model to cluster different image patches into different groups and learn dictionary and mapping function for each group.

\subsection{Adaptive Iterations of Different Noise Levels}
For real image denoising, we can perform well on images which have similar noise levels with the training dataset. How can we deal with the real noisy images whose noise levels are higher than the training dataset? The answer is to remove the noise by more iterations. The input image of each iteration is the recovered image of previous iteration. This makes sense since we can still view the recovered image as a real noisy image. 

This will also bring a second problem, that how we could automatically terminate the iteration. This can be solved by two methods. One way is to compare the images between two iterations and calculate their difference, the iteration can be terminated if the difference is smaller than a threshold. The other way is to estimate the noise level of the current image and terminate the iterations when the noise level is lower than a preset threshold. We employ the second way and set the threshold as 0.0001 in our experiments. In fact, most of our testing images will be denoised well in one iteration.

\subsection{Efficient Model Selection by Gating Network}
In the Gaussian component selection procedure, if we employ the full posterior estimation, the speed is not fast. Our algorithm can be speeded up by introducing the Gating network model.

%------------------------------------------------------------------------
\section{Experiments}

We compare with popular software NeatImage which is one of the best denoising software available. All these methods need noise estimation which is vary hard to perform if there is no uniform regions are available in the testing image. The NeatImage will fail to perform automatical parameters settings if there is no uniform regions.

\subsection{Parameters}
We don't fine tune the parameters both in the training and testing datasets.

\subsection{Real Image Denoising}
We compare the proposed method with the famous BM3D \cite{bm3d} and WNNM \cite{wnnm}, Cascade of Shrinkage Fields (CSF) \cite{csf}, trainable reaction diffusion (TRD) \cite{chen2015learning}, plain neural network based method MLP \cite{burger2012image}, the blind image denoising method Noise Clinic \cite{noiseclinic}, and the commercial software Neat Image. The RGB images are firstly transformed into YCbCr channels and restored by these methods. Then the denoised RGB image is obtained by transforming the restored YCbCr image back.

We evaluate the competing denoising methods from various research directions on two datasets. Both the two datasets comes from the \cite{crosschannel2016}. The first contains 3 cropped images of size $512\times512$. The other dataset contains 42 images cropped to size of $500\times500$ from the 17 images provided in \cite{crosschannel2016}. The 60 images contain most of the scenes in the 17 images \cite{crosschannel2016}.
\begin{table*}
\caption{Average PSNR(dB) results of different methods on 3 real noisy images captured by Canon EOS 5D mark3 at ISO3200 in \cite{crosschannel2016}.}
\label{tab1}
\begin{center}
\renewcommand\arraystretch{1}
\begin{tabular}{|c||c|c|c|c|c|c|c|c|c|c|}
\hline
Image & \textbf{Noisy} &\textbf{BM3D}&\textbf{WNNM}&\textbf{CSF}&\textbf{TRD}&\textbf{MLP}& \textbf{Noise Clinic}& \textbf{Neat Image}&\textbf{Ours}
\\
\hline
1& 37.00 & 37.08 & 37.09 &  37.46  &  37.51  &  32.91  & \textbf{ 38.76}  & 37.68   & 38.63  
\\
\hline
2& 33.88 & 33.95  &  33.95  &  34.90  &  35.04  & 31.94   &  35.69  &  34.87  & \textbf{ 35.96 }
\\
\hline
3& 33.83  & 33.85  & 33.85   & 34.15   &   34.07 & 30.89   & \textbf{35.54 }  &  34.77  &  35.51 
\\
\hline
Average & 34.90  &  34.96 &  34.96  & 35.50   & 35.54   &  31.91  &  36.67  &  35.77  &  \textbf{ 36.70}
\\
\hline
\end{tabular}
\end{center}
\end{table*}

\begin{table*}
\caption{Average SSIM results of different methods on 3 real noisy images captured by Canon EOS 5D mark3 at ISO3200 in \cite{crosschannel2016}.}
\label{tab1}
\begin{center}
\renewcommand\arraystretch{1}
\begin{tabular}{|c||c|c|c|c|c|c|c|c|c|c|}
\hline
Image & \textbf{Noisy} &\textbf{BM3D}&\textbf{WNNM}&\textbf{CSF}&\textbf{TRD}&\textbf{MLP}& \textbf{Noise Clinic}& \textbf{Neat Image}&\textbf{Ours}
\\
\hline
1& 0.9345 & 0.9368  & 0.9372  & 0.9599   &  0.9607  & 0.9043   &  0.9689  & 0.9600   &\textbf{ 0.9712  }
\\
\hline
2& 0.8919 &  0.8848 &  0.8951 & 0.9159   &  0.9187  &  0.8498  &  0.9427  &  0.9308  & \textbf{ 0.9434 }
\\
\hline
3& 0.9128  & 0.9136 & 0.9136  & 0.9254   &  0.9279  &  0.8635  &  0.9476  & 0.9463   & \textbf{ 0.9529 }
\\
\hline
Average &  0.9131  & 0.9117 &  0.9153 & 0.9337   & 0.9358   &  0.8725  & 0.9531   & 0.9457   &  \textbf{0.9558} 
\\
\hline
\end{tabular}
\end{center}
\end{table*}




\begin{table*}
\caption{Average PSNR(dB) and SSIM results of different methods on 42 cropped images from 17 real noisy images in \cite{crosschannel2016}.}
\label{tab1}
\begin{center}
\renewcommand\arraystretch{1}
\begin{tabular}{|c||c|c|c|c|c|c|c|c|c|c|}
\hline
Measure & \textbf{Noisy} &\textbf{BM3D}&\textbf{WNNM}&\textbf{CSF}&\textbf{TRD}&\textbf{MLP}& \textbf{Noise Clinic}& \textbf{Neat Image}&\textbf{Ours}
\\
\hline
PSNR& 34.36 & 34.36 & 34.40 & 36.11 & 36.05 & 34.41 & 37.68 & 36.58 & 36.15
\\
\hline
SSIM& 0.8552 & 0.8553 & 0.8577 & 0.9215 & 0.9211 & 0.9012 & 0.9470 & 0.9145 & 0.9236
\\
\hline
\end{tabular}
\end{center}
\end{table*}


\section{Conclusion and Future Work}

In the future, we will evaluate the proposed method on other conputer vision tasks such as single image super-resolution, photo-sketch synthesis, and cross-domain image recognition. Our proposed method can be improved if we use better training images, fine tune the parameters via cross-validation. We believe that our framework can be useful not just for real image denoising, but for image super-resolution, image cross-style synthesis, and recognition tasks. This will be our line of future work.

{\small
\bibliographystyle{unsrt}
\bibliography{egbib}
}

\end{document}
